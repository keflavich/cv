\documentclass[12pt]{article}

%\usepackage{scimisc-cv}

\usepackage{url}
\usepackage{pbox}
\usepackage{hyperref}
\usepackage[utf8]{inputenc}
\usepackage{listliketab}
\usepackage{enumitem}

\usepackage[margin=1.0in,left=1.0in,right=1.0in]{geometry}

%\usepackage{natbib}  % Requires natbib.sty, available from http://ads.harvard.edu/pubs/bibtex/astronat/

\usepackage[numbers]{natbib}

%\usepackage[square,sort,comma,numbers]{natbib}
%\usepackage[sorting=ydnt]{biblatex}
%\setlength{\bibsep}{0pt}
%\citestyle{aa}  % (Author YYYY) references instead of (Author, YYYY)
\newcommand\apj{{ApJ}}%
\newcommand\apjl{{ApJ}}%
\newcommand\aj{{AJ}}%
\newcommand\aap{{A\&A}}%
\newcommand\apss{{Ap\&SS}}%
\newcommand\araa{{ARA\&A}}%
\newcommand\nat{{Nature}}%
\newcommand\aapr{{A\&A~Rev.}}%
\newcommand\mnras{{MNRAS}}%
\newcommand\apjs{{ApJS}}%
\newcommand\pasp{{PASP}}%
\newcommand\baas{{BAAS}}%
\newcommand\ssr{{SSR}}%
\usepackage{titlesec}
%\titleformat{\section}{\large\bfseries}{\thesection}{}{}
\pagestyle{empty} % no page numbers 
\newcommand{\myname}[1]{\textbf{{\normalsize #1}}}

%\usepackage{fancyhdr}
%\pagestyle{fancy}
%\fancyhf{}


\usepackage{totcount}

\newtotcounter{citnum} %From the package documentation
\def\oldbibitem{} \let\oldbibitem=\bibitem
\def\bibitem{\stepcounter{citnum}\oldbibitem}

%\usepackage{xspace}
\newcommand{\nrefereed}{143\xspace}
\newcommand{\ntotalpapers}{381\xspace}
\newcommand{\ncites}{15459\xspace} % refereed
\newcommand{\ncitestotal}{16020\xspace}
\newcommand{\hindex}{43\xspace}
\newcommand{\nfirst}{20\xspace}
\newcommand{\ncitesfirst}{1011\xspace}

\usepackage{xspace}
\newcommand{\nrefereed}{143\xspace}
\newcommand{\ntotalpapers}{381\xspace}
\newcommand{\ncites}{15459\xspace} % refereed
\newcommand{\ncitestotal}{16020\xspace}
\newcommand{\hindex}{43\xspace}
\newcommand{\nfirst}{20\xspace}
\newcommand{\ncitesfirst}{1011\xspace}


\renewcommand{\refname}{Refereed Publications as of \today\\
\nfirst first author, \total{citnum} total, with \ncitestotal\ citations:}

\usepackage{multibbl}
%\newcites{biba,bibb}{Bib A, Bib B}
\newbibliography{biba}
\newbibliography{bibb}

%\renewcommand{\refname}{Refereed Publications closely related to proposal}
%as of \today\\
%\nfirst first author, \total{citnum} total, with \ncitestotal\ citations:}

\begin{document}

%\begin{minipage}{\textwidth}

%\noindent {\Large \textbf{Biographical Sketch}}

\begin{center}
{\large Dr. Adam G. Ginsburg}\\
Assistant Professor, University of Florida\\
Bryant Space Science Center,
1772 Stadium Road,
Gainesville, FL 32611\\
E-mail: adamginsburg@ufl.edu | Web: \url{www.astro.ufl.edu/people/faculty/adam-ginsburg/} %/ adam.g.ginsburg@gmail.com
%\par ORCID: 0000-0001-6431-9633
% why is the hspace hack necessary?  because CENTER, not CENTERING
%\par Website: \url{www.adamgginsburg.com}
\end{center}



%\vfill  
%\nopagebreak

\section{Research Interests and Associated Activities}
PI Ginsburg has led the data reduction efforts for several large programs, including
the Bolocam and MUSTANG Galactic Plane Surveys.  He is co-PI of the ALMA Large Programs
ALMA-IMF, which aims to measure the core mass function in the most actively star-forming
parts of our Galaxy, and ACES, the ALMA Central Molecular Zone Exploration Survey.
He is PI of several NSF grants aimed at exploring the relation of high-mass star formation
and the initial mass function.
He is PI of a JWST program to study star formation and ices in the Galactic center.
His discovery of salt around high-mass young stellar objects has opened a new field of
astromineralogy that is beginning to be explored with ALMA.
He leads significant software development and is on the astropy core
development team and is lead developer of the astroquery archive querying tool.  
He has pioneered the use of protostar-counting measurements of the star and cluster
formation efficiency in the Galaxy's highest-mass clouds.
Ginsburg leads a vibrant young group at UF, with six graduate students, one postdoc,
and several undergraduates involved in the observations and theory of high-mass
star formation.

\section{Professional Preparation: }
\begin{tabular} {ll}
    2013~PhD Astrophysics & University of Colorado, Boulder, Colorado \\
    2009~M.S. Astrophysics & University of Colorado, Boulder, Colorado \\
    2006~B.S. Astrophysics & Rice University, Houston, Texas \\
\end{tabular}


%%\vfill  
%\nopagebreak


\setlength{\extrarowheight}{2pt}
\section{Appointments:}
\begin{listliketab}
    %\storetstyleof{itemize}
    % total 7in
    \begin{tabular}{p{0.8in}p{6.2in}}
    2023 -      & Associate Professor, University of Florida,  Gainesville, Florida \\
    2019 - 2023 & Assistant Professor, University of Florida,  Gainesville, Florida \\
    2016 - 2019 & Jansky Fellow, National Radio Astronomy Observatory, Socorro, New Mexico \\
    2013 - 2016 & ESO Fellow, European Southern Observatory, Garching, Germany \\
    2007 - 2013  & Graduate Research Assistant, Center for Astrophysics and Space Astronomy,\\
     & University of Colorado, Boulder, CO \\
%    2010 - 2013 & Instructor  & Department of Astrophysical and Planetary Sciences, \\
%                           && University of Colorado, Boulder, CO \\
%    2007 - 2011    & Teaching Assistant & Department of Astrophysical and Planetary Sciences, \\
%                                    && University of Colorado, Boulder, CO \\
%    2007 & Research Assistant & Department of Physics and Astronomy, \\
%                                          && University of Denver, Denver, CO \\
    \end{tabular}
\end{listliketab}

%\end{minipage}
%\begin{minipage}{\textwidth}

%\section*{Areas of Research: }
\begin{itemize}
    \item The astrophysics of massive star formation and the processes
        governing the stellar initial mass function.
    \item The physical properties of the molecular interstellar medium,
        supersonic turbulence, and developing molecular probes
        of local physical conditions. 
    \item Multiwavelength observing with radio interferometers and single dish
        instruments and near-infrared imaging and spectroscopy
    \item The development of astronomical software tools, especially for large
        data cubes and archival data access.
\end{itemize}


%\vfill  
%\nopagebreak


%\clearpage
\setlength{\extrarowheight}{0pt}
\section*{Honors/Awards: }
%\vspace{-12pt}
\begin{tabular}{ll}
     2024 & Waldo W. Neikirk Professorship \\
     2016 & National Radio Astronomy Observatory Jansky Fellowship \\
     2013 & European Southern Observatory Garching Postdoctoral Fellowship \\
     2011 & University of Colorado Chance Irick Cooke Fellowship for Excellence in Research \\
     2010 & NRAO Green Bank Student Observing Support (\$35,000) \\
     2010 & NSF GRFP Honorable Mention \\
     2009 & NSF GRFP Honorable Mention \\
     2008 & NSF GRFP Honorable Mention \\
     2008 & NRAO Photo Contest First Prize (\$1000)\\
     2008 & University of Colorado Astrophysical and Planetary Sciences Excellence in Teaching award  \\
     2006 & National Radio Astronomy Observatory - summer REU with David Meier  \\
\end{tabular}


%\vfill  
%\nopagebreak





%\section{Publications (closely related): }
%\nocite{Ginsburg2019a}
\nocite{biba}{Ginsburg2023}
\nocite{biba}{Ginsburg2022a}
\nocite{biba}{Ginsburg2022b}
%\nocite{biba}{Anderson2020}
%\nocite{biba}{Rivera-Soto2020}
\nocite{biba}{Ginsburg2020}
\nocite{biba}{Rosen2020}
\nocite{biba}{Bally2020}
%\nocite{biba}{Wright2020}
\nocite{biba}{Schworer2019}
\nocite{biba}{Maud2019}
\nocite{biba}{Barnes2019}
\nocite{biba}{Ginsburg2019a}
\nocite{biba}{Ginsburg2019b}
\nocite{biba}{Ginsburg2019c}
\nocite{biba}{Ginsburg2019d}
\nocite{biba}{Leroy2018}
\nocite{biba}{Rosolowsky2018}
%\nocite{biba}{Galvan-Madrid2018}
% \nocite{biba}{Ginsburg2018a}
% \nocite{biba}{Ginsburg2018b}
% \nocite{biba}{Ginsburg2017a}
% \nocite{biba}{Ginsburg2016a}
% \nocite{biba}{Ginsburg2016b}
% \nocite{biba}{Ginsburg2013b}
% \nocite{biba}{Ginsburg2012a}

%\begin{footnotesize}
\bibliographystyle{biba}{apj_revchron_3auth}
%\bibliographystyle{bibb}{apj_revchron}
%\bibliographystyle{apj_revchron}
%\bibliographystyle{nature}
%\bibliography{biba}{cv_cites}{Products \& Publications (closely related):}
%\bibliography{cv_cites}

%\end{footnotesize}

%\begin{footnotesize}
\bibliography{biba}{cv}{Selected Publications}
%\end{footnotesize}


\vfill  
\nopagebreak



%\begin{minipage}{\textwidth}
\setlength{\extrarowheight}{0pt}
\section*{Publications Summary: }
\vspace{-12pt}
\begin{itemize}
\itemsep-3pt
%\item \ntotalpapers ADS entries with \ncitestotal citations 
\item \nrefereed refereed publications with \ncites citations\\
{\normalsize(for a full list, see http://www.adamgginsburg.com/cv.htm)}
\item \nfirst first-author refereed publications with \ncitesfirst citations
\item h-index \hindex

\end{itemize}


\end{minipage}
\vspace{4mm}

%\end{minipage}


%\begin{minipage}{\textwidth}
%\end{minipage}
% \setlength{\extrarowheight}{4pt}
\section*{Research Advising (PhD): }
\vspace{-12pt}
\begin{tabular}{lp{2.1in}p{2.6in}}
    Student & Date  \& Program &   Project \\
    \hline
    Savannah Gramze     &  PhD 2021-2026 (expected) &                                     Gas infall along the Milky Way's bar \\
    Nazar Budaiev       &  PhD 2020-2025 (expected) &                                    The YSO population of Sgr B2 seen through masers and long-baseline ALMA observations  \\
    Alyssa Bulatek      &  PhD 2020-2025 (expected) &                                    Which lines trace what processes in the Galactic Center ISM?  \\
    Theo Richardson     &  PhD 2019-2024 (expected)  &                                   Better understanding of the CMF$\rightarrow$IMF through population modeling  \\
    Desmond Jeff        &  PhD 2019-2023 (expected)  &                                    Star Formation, Hot Cores, and the CMF in Sgr B2 DS \\
    Natalie Butterfield &  PhD at U. Iowa / NRAO Reber Fellow 2017-2018   &                     Cloud Kinematics and Geometry in the Central Molecular Zone \\
    Anna Faye McLeod    &  Ludwig-Maximilian University / ESO PhD Thesis Student 2013-2016  &  FUSION: Comparison of hydrodynamic simulations and observations in nearby high mass star forming regions  \\
\end{tabular}


\section*{Research Advising (Undergraduate): }
\vspace{-12pt}
\begin{tabular}{p{0.85in}p{1.3in}lp{2.6in}}
    Date  & Program & Student &  Project \\
    \hline
    Summer 2022 & Undergraduate &                               Allan Petre    &  VLA observations of W43 for the ALMA-IMF project \\
    Summer 2021, 2022 & Undergraduate &                         Brice Tingle    &  ALMA-IMF-SPICY: SED fitting of YSOs \\
    Summer 2021 & Undergraduate &                               Morgan Himes    &  ALMA-IMF-SPICY: SED fitting of YSOs \\
    Summer 2021 & UF REU &                                      Sydney Petz    &  ALMA-IMF-SPICY: SED fitting of YSOs \\
    Spring 2021 & Undergraduate &                               Aden Dawson    &  JHK H2 imaging of W51 with GTC \\
    Fall 2020- & Undergraduate &                               Michael Fero    &  Modeling Paschen Alpha emission from the Galaxy \\
    Fall 2020 & Undergraduate &                               Parker Ormonde    &  JHK H2 imaging of W51 with GTC \\
    Fall 2020 & Undergraduate &                               Diana Lutz    &  VLA imaging of ALMA-IMF targets to measure free-free contributions \\
    Summer 2020 & UF REU &                               Danielle Bovie    &  Fragmentation structure of W51 with ALMA \\
    Spring 2020- & Undergraduate &                               Madeline Hall    &  Kinematic structure of Sgr B2 \\
    Fall 2019- & Undergraduate &                               Derod Deal    &  Ammonia Masers in W51 \\
    Summer 2019 & NRAO REU Student &                                              Josh Machado    &  Ammonia Temperature Mapping of W51 \\
    Summer 2019 & NRAO REU Student &                                              Tiffany Christian    &  Measuring Mass Functions: Statistical Uncertainties \\
    Summer 2018 & Google Summer of Code &                                         Sushobhana Patra    &  Improving astropy-regions: CRTF and FITS region formats \\
    Summer 2018 & NRAO REU Student &                                              Connor McClellan    &  The YSO population of W51 at high resolution \\
    Summer 2018 & NRAO REU Student &                                              Justin Otter    &  Disks and YSOs in Orion at high angular resolution \\
    Summer 2017 & NRAO Summer Student &                                              Virginie Montes    &  The ionized jet IRAS 16562-3959 \\
    Summer 2017 & NRAO REU Student &                                                 Terry Melo    &  A symmetric ionized and molecular jet in W51 \\
    Summer 2015 & ESO Summer Student &                                                 Dinos Kousidis    &  Merging \texttt{astropy} tools into \texttt{pyspeckit} \\
    Summer 2014 & Google Summer of Code &                                                 Simon Liedtke    &  New tools for \texttt{astroquery}: XMatch, SkyView, Atomic Line List \\
    Summer 2013 & Google Summer of Code &                                                 Madhura Parikh   &  A coherent API for \texttt{astroquery}, a python web database query toolkit \\
\end{tabular}


% 
% \vfill  
% 
% \begin{minipage}{\textwidth}
\setlength{\extrarowheight}{4pt}
\section*{Teaching: }
\vspace{-6pt}

    \textbf{\underline{University of Florida}}

\begin{tabular}{ll}
    Date         & Course \\% & Website \\
    \hline
     Fall 2022    &   AST 4723: Observational Techniques II \\
     Spring 2022  &   AST 6939: Graduate Star Formation \\
     Fall 2021    &   AST 3722: Observational Techniques I  \\%& \url{https://github.com/keflavich/AST3722_Public_Materials}\\
     Spring 2021  &   AST 3722: Observational Techniques I  \\%& \url{https://github.com/keflavich/AST3722_Public_Materials}\\
     Fall 2020    &   AST 4723: Observational Techniques II \\%& \url{https://github.com/keflavich/AST4723-Public-Materials}\\
     Spring 2020  &   AST 3722: Observational Techniques I  \\%& \url{https://github.com/keflavich/AST3722_Public_Materials}\\
     \hline
\end{tabular}

I updated the curriculum of the Obs. Tech. courses to use python for data reduction and analysis; 
the websites with those materials are \url{https://github.com/keflavich/AST3722_Public_Materials}
and \url{https://github.com/keflavich/AST4723-Public-Materials}.  I added a radio astronomy module to 
AST4723, in which students check out 1.5m radio dishes to obtain HI 21cm spectra of the Galaxy.

    \vspace{12pt}
\textbf{\underline{University of Colorado}}

\begin{tabular}{ll}
     Date         & Course \\
     \hline
     Spring 2013  &   Instructor of ASTR 2600: Introduction to Programming for Astronomers (in IDL) \\
     Fall 2012    &   Instructor of ASTR 2600: Introduction to Programming for Astronomers (in IDL) \\
     Summer 2010  &   Co-Instructor of ASTR 1020: Stars and Galaxies \\
     % Fall 2011    &   Co-Instructor of ASTR 6000: Graduate Seminar on the Interstellar Medium \\
     % Fall 2011    &   Teaching Assistant for ASTR 3510: Astronomical Observing (imaging) \\
     % Spring 2010  &   Teaching Assistant for ASTR 3520: Astronomical Observing (spectroscopy) \\
     % Fall 2009    &   Teaching Assistant for ASTR 3510: Astronomical Observing (imaging) \\
     % Fall 2008    &   Teaching Assistant for ASTR 3520: Astronomical Observing (spectroscopy) \\
     % Spring 2008  &   Teaching Assistant for ASTR 3510: Astronomical Observing (imaging) \\
     % Fall 2007    &   Teaching Assistant for ASTR 3520: Astronomical Observing (spectroscopy) \\
     \hline
\end{tabular}

\end{minipage}
\vspace{4mm}

% 
% \vfill  
% 
% 
\begin{minipage}{\textwidth}
\setlength{\extrarowheight}{0pt}
\section*{Conferences and Workshops hosted: }
\vspace{-12pt}
\begin{tabular}{cll}
    Date & Meeting Name & Role  \\
    \hline
    2022 & Cells to Galaxies Talk Series \& Conference  & SOC member \\
    2021 & CMZOOM Talk Series  & Lead organizer \\
    2016 & Lorentz Center workshop ``Apples-to-Apples'':  & Co-organizer \\
         & Comparing simulations \& observations & \\
    2015 &      ESO Central Molecular Zone workshop (2 days)   & Organizer \\
    2015 &      Florence Simulation-Observation Workshop (5 days) & Organizer \\
    2014 &      Workshop on the APEX CMZ 1 mm survey at MPIfR Bonn (1 day)   & Organizer \\
    2014 & ALMA Postdoc Symposium, Tokyo & Co-organizer \\
\end{tabular}
\end{minipage}
\vspace{4mm}

% 
% \vfill  
% 
% %\setlength{\extrarowheight}{4pt}
\section*{Selected Conferences and Workshops attended: }
\vspace{-12pt}
\begin{tabular}{cp{1.8in}p{1.5cm}p{3.0in}}
    Date & Meeting Name & Role & Talk or Poster Title \\
                \hline
    2017 &      Piercing the Galactic Darkness & Invited Talk & Star Formation in the Central Molecular Zone \\
    2017 &      Behind the Curtain of Dust II & Talk & High-mass Star Formation in the Galaxy \\
    2017 &      Multi-Scale Star Formation & Talk & The effects and importance of feedback on high-mass star formation within massive clusters \\
    2016 &      The Local Truth: Star-Formation and Feedback in the SOFIA Era & Talk & Feedback and Accretion around proto-O-stars \\
    2016 &      Half a decade of ALMA: Cosmic Dawns Transformed & Talk & Feedback and Accretion Toward Proto-O-Stars at ALMA's Highest Resolution \\
    2016 &      Sexten: The Role of Feedback in Star Cluster Formation and Evolution  & Talk & The ineffectiveness of feedback in a nearby forming massive cluster, W51 \\
    2016 &      The Early Phase of Star Formation 2016 & Talk & The effects and extent of feedback on dense prestellar gas near proto-OB stars \\
    2016 &      From Stars to Massive Stars & Invited Talk & High-mass Stars and Cores in Massive Protoclusters \\
    2016 &      APEX Ringberg 2016 & Talk & Dense gas in the Central Molecular Zone is warm and turbulent \\
    2015 &      The 6th Zermatt ISM Symposium & Talk & Dense gas in the Central Molecular Zone is warm and heated by turbulence \\
    2015 &      Astropy Lorentz Center Workshop (5 days) & Talks \& unconferences & radio-astro-tools, astroquery, and spectral-cube \\ 
    2015 &      University of Munich Filaments Workshop (3 days) & Talk & W51: The most active star-forming complex in the Galaxy \\
    2015 &      Soul of High Mass Star Formation, Chile & Talk & The Density Structure of the W51 GMC \\
    2014 &      ALMA Arc Node Retreat  & Talk & ALMA's first look at the extended Sgr B2 Cloud \\
    2014 &      Sexten Workshop: The Formation of Globular Clusters  & Talk & The Galactic population of young massive clusters \\
    2014 &      Sexten Workshop: The assembly of massive clusters  & Talk & The density of W51 and its protoclusters \\
    2014 &      Early Phase of Star Formation (EPoS 6)  & Talk & The density structure of The Brick \\
    2014 &      Early Phase of Star Formation (EPoS 6)  & Poster & The density structure of the W51 Giant Molecular Cloud \\
    2013 &      ISM Physical Processes in Garching  & Poster & A measurement of the turbulence driving parameter \\
    2013 &      .Astronomy 5  & Talk & Astroquery: A toolkit for remote data access in python \\
%    2013 &     IAU 303: The Galactic Center  & Poster&  \\
    2013 &      AAS 221  & Thesis Talk & Surveying massive star formation in the Galactic Plane \\
    2012 &      Galactic Scale Star Formation  & Poster& There are no starless massive proto-clusters in the first quadrant \\
    2012 &      Labyrinth of Star Formation  &  Talk& Surveying Pre-Stellar Gas with the BGPS (with an emphasis on what we don't see) \\
%    2011 &      Milky Way  & Talk& The Bolocam Galactic Plane Survey \\
%    2010 &      Stars to Galaxies  & Poster& Star Formation in Perseus Arm Complexes \\
%    2010 &      AAS 217  &  Poster& Formaldehyde Densitometry of Dust Clumps: The shapes and densities of massive star forming regions \\
%    2009 &      AAS 215  &  Poster& The Bolocam Galactic Plane Survey: Data, Early Results, and Future Directions \\
%    \textbullet & IRAM single-dish summer school  &2009&   \\
%    \textbullet & VLA synthesis imaging summer school and summer REU &2006&  \\
\end{tabular}


% 
% % \section{Selected Talks:}
% % \begin{tabular}{cp{1.8in}cp{3.5in}}
% %     \textbullet & ESO Lunch Talk & 2013 & Examining Massive Cluster Formation with H2CO in W51 \\
% %     \textbullet & MPIfR Lunch Talk & 2013 & Surveying Star Formation in the Galactic Plane  \\
% %     \textbullet & CfA Lunch Talk & 2013 & Surveying Star Formation in the Galactic Plane  \\
% % \end{tabular}
% 
% %\setlength{\extrarowheight}{7pt}

\section*{Selected telescope time allocations as PI (2015-):}
\begin{tabular}{p{0.75in}p{3.25in}p{0.65in}p{1.10in}}
                Telescope  & Title & Time & Status \\
    \hline 
    {\textbf{VLA    }\newline {\small 2018} } & VLA/19A-254: Disks and Outflows around O-type stars in W51 & 15 hours & re-Approved \\
    {\textbf{ALMA   }\newline {\small 2018} } & Cycle 6: 2018.1.00057.S: Probing low-mass star formation in the CMZ in Sgr B2 Deep South & 14 hours & re-Approved \\
    {\textbf{GBT    }\newline {\small 2018} } & GBT18A-014: MUSTANG Galactic Plane survey pilot: Protoclusters \& Massive Stars & 31 hours & Approved, \mbox{partly Observed} \\
    {\textbf{VLA    }\newline {\small 2018} } & VLA18A-229: Characterizing high-mass protostars in the whole of Sgr B2 & 36 hours & Observed \\
    {\textbf{ALMA   }\newline {\small 2017} } & Cycle 5: 2017.1.01335.L (co-PI): ALMA-IMF: ALMA transforms our view of the origin of stellar masses & 64 hours & Approved \\
    {\textbf{ALMA   }\newline {\small 2017} } & Cycle 5: 2017.1.00293.S: Characterizing the accretion structures around the HMYSOs in W51 & 8 hours & Approved \\
    {\textbf{ALMA   }\newline {\small 2017} } & Cycle 5: 2017.1.00114.S: Probing low-mass star formation in the CMZ in Sgr B2 Deep South & 14 hours & Approved, \mbox{partly Observed} \\
    {\textbf{ALMA   }\newline {\small 2017} } & Cycle 5: 2017.1.00008.S: The core mass function and its evolution in an extreme protocluster & 10 hours & Approved, \mbox{partly Observed} \\
    {\textbf{GBT    }\newline {\small 2016} } & GBT17A-195: MUSTANG Galactic Plane survey pilot: Protoclusters \& Massive Stars & 31 hours & Approved, \mbox{observed~as} \mbox{GBT18A-014}  \\
    {\textbf{VLA    }\newline {\small 2016} } & VLA16B-202: Disks and Outflows around O-type stars in W51 & 16 hours & Approved, \mbox{partly Observed} \\
    {\textbf{ALMA   }\newline {\small 2016} } & Cycle 4: 2016.1.00620.S: The core mass function and its evolution in an extreme protocluster & 10 hours & Approved, \mbox{partly Observed} \\
    {\textbf{ALMA   }\newline {\small 2016} } & Cycle 4: 2016.1.00550.S: (How) do very massive stars form in our Galaxy? & 7.5 hours & Observed \\
    {\textbf{ALMA   }\newline {\small 2015} } & Cycle 3: 2015.1.00262.S: Digging for rusty bullets at an explosion site & 1.9 hours & Observed \\
    {\textbf{GBT    }\newline {\small 2015} } & GBT/15B-129: Measuring the gas density along the CMZ dust ridge & 13.5 hours & Approved, \mbox{never observed} \\
    {\textbf{ATCA   }\newline {\small 2015} } & C3045: Geometry of clouds and HII regions in the CMZ using H2CO & 84 hours & Published 2015A\&A...584L...7G \\
    \hline 
\end{tabular}

\section*{Selected telescope time allocations as PI (2009 - 2014):}
\begin{tabular}{p{0.75in}p{3.25in}p{0.65in}p{1.10in}}
                Telescope  & Title & Time & Status \\
    \hline 
    {\textbf{VLA    }\newline {\small 2014} } & VLA15A-164: Studying turbulence through the atomic-to-molecular transition & 3.3 hours & Observed \\
    {\textbf{GBT    }\newline {\small 2014} } & GBT14A-329: MUSTANG Galactic Plane survey: HCHIIs in the brightest massive proto-clusters (resubmitted as GBT17A-195) & 14 hours & Approved, \mbox{observed~as} \mbox{GBT18A-014} \\
    {\textbf{ALMA   }\newline {\small 2014} } & Cycle 2: 2013.1.00308.S: Gas temperature and kinematics as key inputs for star formation theory: Cores and turbulence in the massive protocluster W51 & 2.4 hours & Published: 2017ApJ...842...92G \\
    {\textbf{ALMA   }\newline {\small 2014} } & Cycle 2: 2013.1.00269.S: Sgr B2 - The Proving Ground for Star Formation Theories & 6 hours & Published: 2018ApJ...853..171G \\
    {\textbf{LOFAR  }\newline {\small 2014} } & Cycle 2: LC2\_006: A search for p-H2CO, a potential EoR contaminant, toward the Galactic Center, W43, W44, W49, and M82. & 8 hours & Observed  \\
    {\textbf{APEX   }\newline {\small 2014} } & H2CO Thermometry of the CMZ to understand its low star formation rate & 250 hours & Published: 2016A\&A...586A..50G \\
    {\textbf{GBT    }\newline {\small 2014} } & GBT14A-110/GBT12B-221: Density Measurements in G0.253+0.016: Pilot program for CMZ H2CO densitometry & 18 hours & Observed  \\
    {\textbf{KPNO   }\newline {\small 2013} } & 2013A-0399: Star formation in the Central Molecular Zone: Massive Outflows in Sgr C & 6 hours & Observed  \\
    {\textbf{EVLA   }\newline {\small 2013} } & 13A/064: Massive stars and ionized gas in the W51 complex & 13 hours,\newline 4 configs & Published: 2016A\&A...595A..27G \\
    {\textbf{Arecibo}\newline {\small 2012} } & A2854: Density Map of the W51 Giant Molecular Cloud complex & 13 hours & Published: 2015A\&A...573A.106G \\
    {\textbf{GBT    }\newline {\small 2010} } & GBT10B-019: Densitometry of young star-forming complexes throughout the Galaxy & 120 hours & Published: 2013ApJ...779...50G \\
    {\textbf{Arecibo}\newline {\small 2010} } & A2584: Densitometry of young star-forming complexes throughout the Galaxy & 60 hours & Published: 2013ApJ...779...50G \\
    {\textbf{GBT    }\newline {\small 2009} } & GBT09C-049:	Measuring the dense gas mass fraction with H2CO absorption & 4 hours & Published: 2011ApJ...736..149G \\
    \hline 
\end{tabular}

% %\input{otherobserving}
% 
% %\section*{Software:}
\vspace{-10pt}
I am an active developer of a large variety of astronomical python software
tools and a contributor to \texttt{astropy} and its affiliates.  My github
profile (\url{github.com/keflavich}) contains a complete list of projects.
Below is a selection of my most popular packages:
%\vspace{-10pt}

\begin{itemize}
\itemsep-3pt
    \item \texttt{astroquery} (\url{https://astroquery.readthedocs.org}):
        a toolkit for querying internet-hosted astronomical databases
    \item \texttt{pyspeckit} (\url{https://pyspeckit.bitbucket.org}): a software suite
        for visualizing and analyzing spectral line and spectral cube
        data
    \item \texttt{spectral-cube} (\url{https://spectral-cube.rtfd.org}): a library for the manipulation
        of radio spectral cube data
    %\item \texttt{radio-astro-tools} (\url{https://radio-astro-tools.github.io}): an
    %    umbrella organization that includes tools for the analysis of radio
    %    data, including ALMA and EVLA spectral cubes
    \item \texttt{pyradex} (\url{https://github.com/keflavich/pyradex}):
        an object-oriented frontend to the popular RADEX radiative transfer code and
        its peers
    \item \texttt{image-registration} (\url{https://github.com/keflavich/image_registration}):
        a package designed to determine and correct the offsets between images containing only
        diffuse emission
    %\item \texttt{FITS-tools} (\url{github.com/keflavich/FITS_tools}):
    %    a collection of tools for slicing and reprojecting FITS images and cubes
    \item \texttt{sdpy} (\url{https://github.com/keflavich/sdpy}):
        ``Single-Dish python'', a package to support single dish heterodyne data
        reduction and build data pipelines
\end{itemize}


% 
% 
\section*{Service:}
\vspace{-10pt}
\begin{itemize}
\itemsep-3pt
        
    \item Organizer of the ``Python Coffee and Tutorial'' series at ESO, 2014-2016
    \item Referee for the following journals:
        \begin{itemize}
            \itemsep-3pt
            \item \textit{Science}
            \item \textit{Nature}
            \item \textit{Astrophysical Journal}
            \item \textit{Astronomy \& Astrophysics}
            \item \textit{Monthly Notices of the Royal Astronomical Society}
            \item \textit{Revista Mexicana de Astronom{\'i}a y Astrof{\'i}sica}
    \end{itemize}
    \item Served on the SOFIA TAC
    \item Panel chair for a recent NASA grant review panel
    \item ESO ALMA Fellow Duties as part of the European ALMA Regional Center.
        Primary duties include software development, maintenance of the
        Quality Assurance Packager software, and regression testing
    %\item Co-organizer of the December 2014 ALMA Postdoc Symposium in Tokyo
    %\item Member of the \texttt{astropy} (\url{astropy.org}) collaboration
    \item Member of the \texttt{montage} (\url{montage.ipac.caltech.edu}) Image
        Mosaic Engine users group
    %\item (former) Member of the CCAT ISM working group
    \item Member of the Next-Generation VLA (NGVLA) high mass star formation
        working group
    \item Member of the SKA Galactic Science
        working group
\end{itemize}

% 
% \end{minipage}

%\section*{Additional Training:}
\begin{itemize}
\itemsep-3pt
    \item UF Center for Teaching Excellence program (Fall 2019)
    \item ESO Fellows Development Program: MBTI (October 8, 2015)
    \item ESO Fellows Development Program: People Skills (June 18, 2015)
    \item ESO Fellows Development Program: Networking (February 17, 2015)
    \item ESO Fellows Development Program: Presentation Skills (July 3, 2014)
    \item ESO Fellows Development Program: Scientific Writing (March 4, 2014)
    \item ESO Fellows Development Program: Project Management (January 28, 2014)
\end{itemize}





%\section{Outreach:}
%    Judge for Denver Metro Science Fair, 2/27/2008
%    Volunteer for middle school science day at the University of Denver, May 2007
%    Volunteer for University of Colorado Astronomy Day, 2008-2011
%    Volunteer for Sommers-Bausch Observatory Open Houses, 2007-
%    Vail Nature Center presentation 7/19/2008
%    Judge at Boulder Country Day science fair 12/10/2008
%    Judge at Centennial Middle School science fair 1/30/2009
%    <a href="outreach.htm"> Public presentations at REI September 2008, 2009</a>
%    Presentation to Boulder Astronomical Society 1/16/2010
%    Winter Park Star Safari, July 2011

%\input{overviewprose}

% \noindent{\large \textbf{Synergistic Activities:}}
% \begin{enumerate}[topsep=0pt,itemsep=-1ex,partopsep=1ex,parsep=1ex]
% %    \item Proposal reviewer for NASA ADP grant program
%     \item Reviewer for the ALMA science review panel
%     \item Member of the \texttt{astropy} core development team
%     \item Software developer for the \texttt{astrodendro} source cataloging package
%     \item Main developer of the \texttt{radio-astro-tools} analysis tool package
% %    \item Referee for several astronomical journals
% %             \textit{Nature},
% %             \textit{Astrophysical Journal},
% %             \textit{Astronomy \& Astrophysics},
% %             \textit{Monthly Notices of the Royal Astronomical Society},
% %             \textit{Proceedings of the Astronomical Society of Japan}, and
% %             \textit{Revista Mexicana de Astronom{\'i}a y Astrof{\'i}sica}
% %
% \end{enumerate}


%\newpage
%\nocite{*}
%%\section{Publications: }
%\begin{footnotesize}
%\bibliographystyle{apj_revchron}
%%\bibliographystyle{nature}
%\bibliography{cv}
%\end{footnotesize}


\end{document}
