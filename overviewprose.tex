\section*{Research Overview}
I study the formation of massive stars and massive clusters.  Massive stars are
responsible for enriching the interstellar medium with heavy elements in the
early universe.  They produce most of the light and energy in the universe.
The locations where these largest stars form are important: when many massive
stars spawn in the same place, their collective feedback - radiation, winds,
and eventual explosions - is far greater than the sum of the individual
components.  Clustered star formation is therefore an essential ingredient in
understanding the formation of both stars and galaxies.

I aim to understand how the gas structure in molecular clouds governs the
formation of the most powerful massive clusters and the future evolution of
those clouds.  We know that molecular clouds are supersonically turbulent, so I
work to understand how that turbulence affects both the formation of clusters
and how we perceive their parent clouds.

I use primarily submillimeter, millimeter, and radio telescopes to identify and
examine physical conditions.  I use different transitions of molecular lines,
especially formaldehyde (H2CO) and carbon monoxide (CO), to determine the
density, temperature, and kinematics of gas in the interstellar medium.
Much of my effort is on using radiative transfer modeling to turn relatively
inexpensive observations into important physical constraints on gas properties.

In order to make this research possible, I actively develop and distribute
software for the analysis of spectral data, molecular lines, dust continuum
data, and statistical distributions.

